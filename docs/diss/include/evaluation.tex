\chapter{Evaluation}

\section{Experimental Setup}
\subsection{Mininet}
\begin{itemize}
  \item Describe the simulation framework used with Mininet - how the simulation scripts are structured.
  \item Describe the system by which a simulation is performed - include a diagram of how simulation scripts and executable binaries are transferred to the simulation, simulations are performed and resulting log files are returned to the host-machine for analysis and plotting.
  \item Describe the system by which tracing is performed.
\end{itemize}

\subsection{Libpaxos}
\begin{itemize}
  \item Short description of Libpaxos library.
  \item Explain how application was written / any modifications necessary to run the same sample application.
\end{itemize}

\section{Steady state behaviour}
\begin{itemize}
   \item Characterise the steady state behaviour of the system - capturing the latency and throughput in a given configuration. Plot against each other to show characteristics.
   \item Comparison in the steady state behaviour between the project implementation and Libpaxos.
\end{itemize}

\section{Configuration sizes}
\begin{itemize}
   \item Treatment of experiments that describe how latency and throughput vary as a function of the number of different nodes in the configuration varies.
      \item Include description of the exact experiment performed and how confidence intervals were calculated.
   \item Include graphs for: latency against different cluster sizes and throughput against different cluster sizes. Include Libpaxos version of this as well.
\end{itemize}

\section{Failure traces}
\begin{itemize}
  \item Experiment pertaining to crashing a replica and observing system behaviour over time.
  \item Describe how data was averaged - including the sample mean and EWMA.
  \item Graph of latency and throughput as a function of time with a specific point a replica is crashed noted.
  \item Similar graph with a replica restoration.
  \item Graphs / on the same plot with behaviour of Libpaxos.
\end{itemize}

\section{Quorum sizes}
\begin{itemize}
  \item ...
  \item ...
\end{itemize}

\section{Summary}
