\chapter{Introduction}

\section{Background}

Distributed systems suffer from a number of possible errors and failure modes. Unreliability is present in the network where messages can be delayed, re-ordered and dropped and processes can exhibit faulty behaviour such as stalling and crashing. The result of this is that distributed systems can end up inconsistent states and even unable to make progress. \\

Consensus is the reaching of agreement in the face of such unreliable conditions. Transaction systems, distributed databases and leadership elections are all applications that require consensus in order to remain consistent. Consensus algorithms provide a means by which to reach agreement across in a distributed system in the face of such unreliability; this is crucial in the design of distributed systems. \\

Paxos is a consensus algorithm first described by Lamport \cite{Lamport:1998:PP:279227.279229} that allows for consensus to be reached under the typical unreliable conditions present in a distributed system. It is a three-phase commit system that relies on processes participating in the \emph{Synod} voting protocol in order to tolerate the failure of a minority of processes. \\

Paxos is used internally in large-scale production systems such as Google's Chubby \cite{Burrows:2006:CLS:1298455.1298487} distributed lock service, where it is used to maintain consistency between replicas. Microsoft's Autopilot \cite{autopilot-automatic-data-center-management} system for data centre management also uses Paxos, again to replicate data across machines. The extreme generality of Paxos allows it to be used as an underlying primitive for various distributed systems techniques. State Machine Replication \cite{Schneider:1990:IFS:98163.98167} is a technique whereby any application that behaves like a state machine can be replicated across a number of machines participating in the Paxos protocol. Likewise, atomic broadcast \cite{Rodrigues:2003:ABA:942591.942742} can be implemented with Paxos as an underlying primitive. \\

Over time Paxos has been extended and modified to emphasise different performance trade-offs. Multi-Paxos is the most typically deployed variant which allows for explicit agreement over a sequence of values. Another example, Fast Paxos \cite{fast-paxos}, is a variant that reduces the number of message delays between proposing a value and it being chosen. More recently, Flexible Paxos \cite{DBLP:journals/corr/HowardMS16} is a variant that {\color{red}relaxes the requirement on same-phase quorums intersecting} in order to improve performance. \\

There are a number of alternative means of reaching consensus. Viewstamped replication \cite{Oki:1988:VRN:62546.62549} is primarily a replication protocol but can be used as a consensus algorithm. Raft \cite{Ongaro:2014:SUC:2643634.2643666} is a modern alternative to Paxos that attempts to reduce the complexity of implementing Paxos. {\color{red}Why didn't we use these though?}


\section{Aims}

The aim of this project was to produce an implementation of the Multi-Paxos variant of the Paxos algorithm to replicate a {\color{red}toy} distributed application. This is the variant that is used most widely in production systems and provides a foundation upon which to use state machine replication to replicate an application. \\

OCaml will be used as the primary development language. {\color{red}Why OCaml?} \\

{\color{red}Evaluation and simulator explanation.}\\


%\section{Aims}

%The aim of this project was to produce an implementation of the Multi Paxos variant of the Paxos algorithm. Multi Paxos is similar to the original algorithm except that it allows a sequence of values to be agreed upon and assumes the temporary stability of elected nodes in the network. ... \\

%Evaluation will take place on a network simulator in order to test the implementation in an environment with properties similar to those assumed to occur in typical distributed systems.

%Talk about testing for consensus.

%Talk about evaluation a bit.